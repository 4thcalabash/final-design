\chapter{结论}\label{chapter_conclusion}
本文实现了一个自动化的检测工具,它可以帮助安卓应用的开发人员对应用进行测试,帮助发现存在内存泄露问题的公开服务以及清单声明的广播接收器,尽量减少应用中的缺陷和错误。

在本文的解决方案中,首先对被测试应用进行反编译,接下来编写了脚本分析应用组件清单,得到公开服务和清单声明的广播接收器的列表,作为被测试对象。之后启动安卓模拟器,将被测试应用、测试驱动应用安装到模拟器上,并启动测试驱动应用来负责执行整个测试任务:对被测试组件进行反复的启动/关闭,绑定/解绑定,注册/解注册操作,最大程度确保内存泄漏问题被实际触发。最后寻找内存泄漏的组件,将被测试应用的堆镜像文件导出到工作站中,基于开源工具开发了检测安卓组件内存泄漏的定制工具,最终导出组件的内存泄漏情况报告,以指导开发者对应用进行进一步的测试和调试工作。

但是本文的工作也存在相应的不足:如在实验的测试环节,由于工作站配置的制约,选取的应用数量和规模都很小,因此测试数据具有较大的偶然性;由于难度较大,本文针对广播接收器的测试中,只选取了在清单中注册的广播接收器进行测试,而没有对动态注册的广播接收器进行测试。

在未来的工作中,针对本文的上述不足大致有两个改进的方向:第一,针对测试不充分的问题,可以将本文的测试方法进行并行化的改进,然后使用更大量的应用进行充分的测试,来统计实验数据,这样会更有说服力。第二,对于动态注册的广播接收器,由于无法在清单中自动分析出这些接收器,且注册的过程相对比较复杂,可以寻找一些新的方式去对这一部分接收器进行测试。