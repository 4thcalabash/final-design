%!TEX TS-program = xelatex
%!TEX encoding = UTF-8 Unicode

%%
%% 使用 njuthesis 文档类生成南京大学本科生毕业论文的示例文档
%% 
%%

%% 
%% 南京大学本科学位论文模板
%% 2018年封面,摘要都发生了变化,本模板由以下2016年模板更改而来:http://haixing-hu.github.io/nju-thesis/

%% 如需Adobe字体请用(默认)
%\documentclass[adobefonts]{njuthesis}
%% MacOS系统请用
%\documentclass[macfonts]{njuthesis}
%% Windows系统请用
\documentclass[winfonts]{njuthesis}
%\usepackage{enumerate}
\usepackage{enumitem}
%% Linux系统请用
%\documentclass[linuxfonts]{njuthesis}
\newcommand\redbf[1]{\textbf{\textcolor{red}{#1}}}
\renewcommand{\today}{\number\year 年 \number\month 月 \number\day 日}
%%%%%%%%%%%%%%%%%%%%%%%%%%%%%%%%%%%%%%%%%%%%%%%%%%%%%%%%%%%%%%%%%%%%%%%%%%%%%%%
% 设置论文的中文封面
% 论文标题
\title{安卓不可见控件内存泄漏的自动检测}

% 论文作者姓名
\author{王冬杨}
% 论文作者学号
\studentid{161250136}
% 导师姓名职称
\supervisor{马骏}
% 导师职称
\supervisortitle{副教授}
% 论文作者院系
\department{软件学院}
% 论文作者专业方向
\major{软件工程}
% 论文作者的年级
\grade{2016级}
% 论文提交日期,需设置年、月、日。此属性可选,默认值为最后一次编译时的日期,精确到日。
\submitdate{\today}

%%%%%%%%%%%%%%%%%%%%%%%%%%%%%%%%%%%%%%%%%%%%%%%%%%%%%%%%%%%%%%%%%%%%%%%%%%%%%%%
% 设置论文的英文封面

% 论文的英文标题
\englishtitle{Thesis paper template}
% 论文作者姓名的拼音
\englishauthor{San Zhang}
% 导师姓名职称的英文
\englishsupervisor{Professor Si Li}
% 论文作者所在院系的英文名称
\englishdepartment{School of Electronic Science and Engineering}
% 论文作者所在学校或机构的英文名称。此属性可选,默认值为``Nanjing University''。
\englishinstitute{Nanjing University}
% 论文完成日期的英文形式,默认最后一次编译的时间
\englishdate{May 20, 2018}
% 专业
\englishinstitute{Electronic Information Science and Technology}
%%%%%%%%%%%%%%%%%%%%%%%%%%%%%%%%%%%%%%%%%%%%%%%%%%%%%%%%%%%%%%%%%%%%%%%%%%%%%%%
% 设置论文的页眉页脚
\usepackage{fancyhdr}
%\usepackage{enumitem}

\pagestyle{fancy}
%\lhead{\bfseries 141180092 }
\chead{安卓不可见控件内存泄漏的自动检测}
\rhead{王冬杨}
%\lfoot{From: K. Grant}
%\cfoot{To: Dean A. Smith}
%\rfoot{\thepage}
\renewcommand{\headrulewidth}{0.4pt}
%\renewcommand{\footrulewidth}{0.4pt}
%%%%%%%%%%%%%%%%%%%%%%%%%%%%%%%%%%%%%%%%%%%%%%%%%%%%%%%%%%%%%%%%%%%%%%%%%%%%%%%

\usepackage{xcolor}
\usepackage{minted}

\usepackage{listings}

\usepackage{color}
\definecolor{gray}{rgb}{0.4,0.4,0.4}
\definecolor{darkblue}{rgb}{0.0,0.0,0.6}
\definecolor{cyan}{rgb}{0.0,0.6,0.6}

\lstset{
	basicstyle=\ttfamily,
	columns=fullflexible,
	showstringspaces=false,
	commentstyle=\color{gray}\upshape
}
\begin{document}

% 制作中文封面
\maketitle
% 制作英文封面
% \makeenglishtitle
% 毕业论文过程管理四页表
% \controlpage %可以将word文件交给老师签字后扫描转成pdf,然后命名为controlpage.pdf

% 论文的中文摘要
\begin{abstract}

在安卓应用中,服务和广播得到了广泛应用,提供诸如下载,数据更新,跨应用通信等功能。但由于开发者经常忽视这些不可见控件的生命周期管理,因此内存泄漏发生的几率很高。

本文将会关注新版本安卓系统(Android 8+)中公开服务(Exported Services)以及静态注册的广播接收器的内存泄露问题,阐述公开服务和广播接收器内存泄露的检测方法,并编写一套供服务开发人员使用的自动分析组件内存泄露的检测工具,最后会从应用市场(App Store)中下载真实的应用,进行内存泄漏检测和分析。

% 同时应该注意到,空白页是故意留白,以便章节开头能够出现在偶数页。
% 中文关键词。关键词之间用中文全角分号隔开,末尾无标点符号。
\keywords{安卓系统;内存泄漏;安卓服务;安卓广播}
\end{abstract}

%%%%%%%%%%%%%%%%%%%%%%%%%%%%%%%%%%%%%%%%%%%%%%%%%%%%%%%%%%%%%%%%%%%%%%%%%%%%%%%
% 论文的英文摘要
% \begin{englishabstract}
% The diversity of handwritten Chinese text make it a promising but % challenging computer vision problem. 
% 英文关键词。关键词之间用英文半角逗号隔开,末尾无符号。
% \englishkeywords{Handwritten Chinese, Text recognition, Deep learning}
% \end{englishabstract}

%%%%%%%%%%%%%%%%%%%%%%%%%%%%%%%%%%%%%%%%%%%%%%%%%%%%%%%%%%%%%%%%%%%%%%%%%%%%%%%
% 论文的前言,应放在目录之前,中英文摘要之后
%
%\begin{preface}
%
%在过去的40年中,手写中文文本领域识别(HCTR)取得了很大的进展[1,2]。
%
%\vspace{1cm}
%\begin{flushright}
%饶安逸\\
%2018年5月15日于南大仙林
%\end{flushright}
%
%\end{preface}

%%%%%%%%%%%%%%%%%%%%%%%%%%%%%%%%%%%%%%%%%%%%%%%%%%%%%%%%%%%%%%%%%%%%%%%%%%%%%%%
% 生成论文目录
\tableofcontents

%%%%%%%%%%%%%%%%%%%%%%%%%%%%%%%%%%%%%%%%%%%%%%%%%%%%%%%%%%%%%%%%%%%%%%%%%%%%%%%
% 生成插图清单。如无需插图清单则可注释掉下述语句。
%\listoffigures

%%%%%%%%%%%%%%%%%%%%%%%%%%%%%%%%%%%%%%%%%%%%%%%%%%%%%%%%%%%%%%%%%%%%%%%%%%%%%%%
% 生成附表清单。如无需附表清单则可注释掉下述语句。
%\listoftables

%%%%%%%%%%%%%%%%%%%%%%%%%%%%%%%%%%%%%%%%%%%%%%%%%%%%%%%%%%%%%%%%%%%%%%%%%%%%%%%
% 开始正文部分
\mainmatter

%%%%%%%%%%%%%%%%%%%%%%%%%%%%%%%%%%%%%%%%%%%%%%%%%%%%%%%%%%%%%%%%%%%%%%%%%%%%%%%
% 学位论文的正文应以《绪论》作为第一章
\chapter{绪论}\label{chapter_introduction}

本章将介绍一些与本文研究相关研究背景,简要的总结前人的相关工作,并指出前人工作的不足之处。进而说明本文的主要贡献,以及本文在实现时面临的几个主要的挑战,最后说明本文的结构。
\section{研究背景}
在安卓应用中,服务(Services)和广播(Broadcast)得到了广泛的使用。服务可以在安卓应用的后台保持长期运行,提供诸如下载、数据更新等重要功能。然而,正因为服务长期运行于后台的特点,使其往往容易产生异常(Errors)。如果服务的编写人员缺少警惕性,服务中出现的错误(Bug)可能会导致诸多问题,严重者可能引起应用崩溃,甚至系统死机;广播是一种常被用来进行跨应用的通信的通信手段,应用在使用广播进行与系统或者与其他应用进行通讯时,应用需要编写广播接收器(Broadcast Receiver)。负责运行广播接收器的时应用的主线程种,因此在广播接收器中不适合进行耗时操作,通常会在广播接收器中启动服务来进行后续的处理,因此广播接收器也可能通过服务或者自身导致内存泄漏。

安卓应用中的内存泄露指资源(内存对象、句柄、服务等)将不再被使用,但却无法被安卓系统的垃圾回收器(Garbage Collector)所回收,同时也是服务中的一大类常见问题。服务如果出现内存泄露,将会导致内存使用量意外大幅度增加,进而使得系统效率降低,严重影响用户体验。

有一类服务被称为\textbf{公开服务},即指定了\textbf{“exported:true”}属性的服务。其他应用在满足一定条件时(满足权限要求等)可以启动应用的公开服务,因此内存泄露的问题将会变得更加复杂。

由于在目前的安卓版本(安卓10)中,安卓操作系统中的“电池优化策略”\cite{pms}会禁止跨应用启动后台服务\cite{android-service-limit},而这种测试方式在早先的安卓版本中(安卓7及更早)是可行的。因此在现有的最新安卓系统(安卓10)中,组件的内存泄漏检测方法与先前的方法\cite{jun2018lesdroid}将会有所差别。

\section{相关工作}

Erika 等人在早先的安卓版本中,编写了一个检测跨应用通信安全问题的工具\textbf{Com Droid}\cite{chin2011analyzing},文中阐述的方法对于跨应用测试具有借鉴意义。
\newline

刘洁瑞等人在2016年,针对安卓系统资源(如相机等)的内存泄漏问题,构造了\textbf{35}组标准测试集并使用\textbf{ResLeakBench}\cite{liu2016}进行了较为广泛的测试,该文所研究的系统资源泄露的场景,对本文研究的组件内存泄漏的场景有一定的借鉴意义。
\newline

在早先的安卓版本中,跨应用启动服务这一行为是被允许的,南京大学的马骏等人实现过一个公开服务(Exported Services)内存泄漏的检测工具LES Droid\cite{jun2018lesdroid},文中采用的方式分为四步:

\begin{enumerate}
	\item 使用apktool反编译工具\cite{apktool}对被测试应用进行反编译,获取被测试应用的安卓组件清单(AndroidManifest.xml)文件,解析获取应用中所有的公开服务的包名和类名。
	\item 将测试驱动应用、被测试应用通过adb安装到模拟器中,启动测试驱动应用。
	\item 测试驱动应用重复启动、关闭被测试的服务,在满足一定测试强度之后,导出被测试应用的堆镜像文件(.hprof files)。
	\item 基于MAT内存分析库\cite{mat}编写堆镜像文件的分析工具,自动检测内存泄漏并统计泄露的入口等。
\end{enumerate}

\label{pre-result}
文中的数据指出:在41537个被测试应用中,共在其中28662(69\%)个应用中检测出74831个服务,其中3934(13.7\%)个应用拥有公开服务。经过去重、安装测试以及应用商店评分筛选,有375个实际测试应用,最终通过不同的测试配置,最终检测到在18.7\%的应用中有16.8\%的服务存在内存泄漏问题。

\section{主要挑战}

本文面临的挑战主要有:

\begin{enumerate}
	\item 如何驱动测试流程自动化进行,对被测试组件进行自动化的创建和销毁。
	\item 寻找办法能解决“电池管理限制”对测试流程和方法带来的多种限制。
	\item 如何对内存文件进行自动化进行分析,找出内存泄漏的组件实例。
\end{enumerate}

\section{现有研究的不足}

在测试的方法上:在安卓系统的升级中,着重推出了“电池管理限制”,而这一限制使得原有的测试方式在最新的安卓版本种无法直接使用,需要针对“电池管理限制”进行相应的修改。

在测试对象上:已有的研究只针对安卓的系统资源,以及安卓的公开服务进行了内存泄漏的测试,对于其他不可见组件(广播接收器等)并没有进行系统的测试。
\section{本文贡献}

本文将会在最新的安卓版本(安卓10)上对服务和广播进行内存泄漏的检测和测试。探索了一种解除电源管理限制的方法。分析实验数据,评价“电池管理限制”带来的影响。

\section{本文主要工作}
本文旨在探索一套适用于安卓10版本的公开服务和静态注册广播接收器的内存泄漏检测方法。主要工作如下:
\newline
\begin{enumerate}
\item 对原有的测试方法进行修改,使得该方法能够适用于安卓10版本。

\item 在桩应用上进行测试,验证自动检测工具的可行性和正确性。

\item 在应用商店中下载真实应用,进行自动化测试分析实验结果。

\end{enumerate}
\section{本文结构}
本文的各章节组织结构如下:
第一章\textbf{绪论\ref{chapter_introduction}}简要的说明了本文的研究对象,和采用的测试方法,并总结了相关论文的工作等。第二章\textbf{背景\ref{chapter_background}}详细介绍了安卓广播和服务的注册、启动、生命周期等基础知识,并且对安卓组件内存泄漏的原理做了举例,还简要介绍了新版本安卓中的“电池管理限制”。第三章\textbf{技术方案\ref{chapter_system}}中具体地解释了自动化检测工具的所有流程和实现细节,并通过追溯安卓源码发现了后台启动服务的方法(后台启动服务默认为不允许的行为)。第四章\textbf{实验\ref{chapter_experiment}}进行了仿真测试与真实测试,仿真测试目的在于验证自动检测工具的正确性和可行性,真实测试从安卓应用市场中下载了真实应用,使用本文的自动检测工具进行了测试。第五章\textbf{结论\ref{chapter_conclusion}}总结了本文的主要工作,并通过实验数据发现在新版本的安卓系统中,应用的组件内存泄漏问题依然较为广泛的存在。
\chapter{背景}\label{chapter_background}
本章将主要介绍与本文研究相关的背景知识:安卓系统,安卓服务,安卓广播接收器。具体的将会介绍这两种组件分别的启动方式、注册方式、生命周期、内存泄漏的成因等。

首先将简单介绍安卓操作系统;接下来重点本文的测试对象:安卓服务以及安卓广播接收器。

\section{安卓系统}
\subsection{发展历史}
\textbf{安卓}是一个基于\textbf{Linux kernel}以及众多开源软件的移动设备操作系统,是一个为触屏设备定制的操作系统。

安卓系统诞生于2013年,由\textbf{安迪鲁宾}等人开发制作,于2015年被\textbf{谷歌(Google)}公司收购,现在由\textbf{谷歌(Google)}公司进行持续的更新迭代和开发。

现在安卓已经以\textbf{Apache免费开放源代码许可}方式进行了开源授权,这一措施使得厂商可以根据自己的设备搭载定制化的修改过的安卓系统,进而大大加速了安卓系统在移动设备上的普及。目前安卓设备已经超越\textbf{微软Windows操作系统},成为全球第一的操作系统。

\subsection{主要组件}

安卓应用中有四大基本组件:
\begin{enumerate}
	\item \textbf{Activity } Activity(即活动)是安卓应用中的图形化界面,在安卓应用中几乎所有与用户的交互都是通过它来完成,是安卓应用中使用最多的一种重要组件,活动之间可以实现切换和通信,从而给用户带来更好的使用体验。
	
	在应用中,系统会有一个活动栈持有所有活动的引用,当新的活动被创建时,将会进入栈的顶部,而当一个活动被销毁时,将会被从栈中弹出,借助活动栈,可以实现页面跳转,回退等操作。
	
	活动共有四种状态:\textbf{运行态},\textbf{暂停态},\textbf{停止态},\textbf{销毁},开发者可以通过生命周期函数对状态进行有效的管理,但活动并非本文的研究对象,因此不做过多赘述。
	
	\item \textbf{Service } Service(即服务)是安卓应用的后台组件,服务并没有可视化的界面,不被用户所感受到,但服务可以在后台为应用提供重要功能,例如数据更新,播放音乐等。
	
	由于活动负责响应用户的交互,因此需要保证活动处于高响应优先级状态,因此例如网络下载,文件读写等耗时的操作需要在不影响主线程的其他线程中完成,这就是服务的存在意义。
	
	服务的状态有两种:\textbf{启动状态}和\textbf{绑定状态}
	
	本文接下来将在\ref{service}章节中对服务进行详细的讲述。
	
	\item \textbf{Broadcast } Broadcast(即广播)也是安卓应用的后台组件,广播的存在主要目的是为跨应用通信和系统与应用通信提供了一种通用的方法。应用可以主动发出广播,也可以被动接收广播,也可以接收来自系统的广播,应用可以注册广播接收器,在接收到广播时进行一系列的响应动作,这就完成了跨应用通信,这使得用户安装的众多应用之间,可以产生联动交互,提升了用户体验。
	
	本文接下来将在\ref{broadcast}章节对广播进行详细的讲述。
	
	\item \textbf{ContentProvider } ContentProvider(即内容提供器)时安卓系统中专门为不同应用之间数据通信的组件,是对\textbf{标准查询语句}的高层封装。应用可以设置部分数据与其他应用共享,这样其他应用通过内容提供器就可以获取到这一部分共享数据。需要注意的是,在使用内容提供器时,也需要对生命周期进行有效的管理,否则容易导致内存泄漏问题,过度占用系统的后台资源。
\end{enumerate}
\section{安卓服务}\label{service}

在安卓系统中,每个安卓应用都对应着一个主线程,这个主线程将负责处理界面计算和渲染、负责处理用户的交互以及负责响应生命周期事件等。这些任务对主线程产生了快速响应的要求,否则会导致用户体验质量的下降。换言之,在主线程中只能进行不耗时(几毫秒级别)的计算和操作,而任何耗时较长的计算和操作都必须在主线程之外的单独的后台线程之上来完成。这样可以使得用户在积极的与应用交互时,应用在后台可以积极的响应和运行。

然而,后台任务需要运行,则不可避免地会占用(消耗)安卓设备的系统资源:例如占用若干内存容量和消耗一定电池电量等。如果后台任务操作不当(如占用大量内存导致系统卡顿,长期进行密集计算导致电池电量下降变快),也可能会导致用户体验下降;同时如果安卓服务的开发人员在开发时引入了\textbf{程序缺陷},使得服务在运行时产生了\textbf{内存泄漏},将会使得前面的现象变得更加严重。因此随着安卓系统的更新,其对服务组件也进行了更多的限制,以尽可能保证用户体验。

\begin{itemize}
	\item \textbf{安卓6.0(API 级别 23) } 此版本的安卓系统种,新增了重要的\textbf{低电量模式}。在关闭安卓设备的屏幕,而且安卓设备的顶位没有发生显著变化的情境下,低电量模式将被启动,对正在运行中的应用行为进行适当的限制,以减少电量的消耗,例如:限制网络访问,限制同步等。即当开启低电量模式时,服务的行为会受到一定程度的限制。
	\item \textbf{安卓7.0(API 级别 24)} 此版本的安卓系统禁止了应用注册隐式广播,取而代之的时,应用必须显式进行广播的注册,此举的目的在于使应用更多关注广播接收器生命周期。另外一个变动在于将上个版本中推出的低耗电模式推广成为一个随时进行的系统耗电优化:即当屏幕一段时间处于未唤醒状态且未接入电源,低耗电模式就会启动,而无需手动开启。即只要设备处于闲置状态时,服务的行为随时都会受到限制。
	\item \textbf{安卓8.0(API 级别 26)} 此版本的安卓系统\textbf{禁止处于后台的应用启动新的服务},取而代之的是需要显式经过用户许可,启动前台服务。这项限制对跨应用测试的研究带来了不便(跨应用测试一般由测试人员编写的测试驱动应用去启动其他(后台)应用的组件)。
	\textbf{安卓9.0(API 级别 28)} 新增了\textbf{应用待机存储分区}。系统将会根据应用的使用模式,给应用动态分配优先级,系统在分配系统资源时,优先级将会是一个重要的指标。
\end{itemize}

本节接下来将会详细介绍服务的启动方式,生命周期,注册方式,以及内存泄漏的成因。

\subsection{服务的启动方式}
安卓应用中的服务可以通过两种方式启动\cite{service}:

\textbf{start 模式 } 通过此方式启动的服务,必须为显式方式:即构造出特定的\textbf{Intent}对象,接着调用\textbf{startService() API}来启动目标服务。

\textbf{bind 模式 } 通过此方式绑定的服务,通过调用\textbf{bindService() API}将目标服务与特定的系统组件绑定。被绑定的服务提供接口供其他组件与之交互。

一个服务可以同时通过以上两种方式启动。

\subsection{服务的生命周期}
服务的生命周期根据启动方式不同分为两种\cite{service}:

\textbf{start 模式 } 通过\textbf{startService() API}启动的服务,在调用\textbf{stopSelf()}方法将自身停止运行前,服务将始终处于运行状态。也可以由第三方构造特定的\textbf{Intent}实例,调用系统的\textbf{stopService(Intent ) API}将服务停止运行。

停止运行的服务将会被\textbf{GC(Garbage Collector)}回收。

\textbf{bind 模式 } 通过\textbf{bindService() API}启动的服务,在其他组件调用\textbf{unbindService() API}解除绑定之前,可以通过\textbf{IBinder}接口与之进行交互,直到。

同一个服务可以同时绑定到多个组件之上,由于服务时单例模式,即同一个服务只能有一个实例在系统种运行,因此若要使得该服务停止运行,并被\textbf{GC}回收,需要满足的条件是:所有组件都解除了绑定。

每个安卓应用都关联一个\textbf{ActivityThread}实例,负责调度和执行该应用的各种组件。\textbf{ActivityThread}有一个\textbf{ArrayMap}类型的成员变量\textbf{mServices},其中保存了所有没有被销毁的服务的引用。一旦某个服务的实例被销毁,其引用将会从\textbf{mServices}中删除。

\subsection{服务的注册方式}
\begin{listing}[htbp]
	\centering
	\caption{服务的注册方式}
	\begin{minted}[encoding=utf8,
	frame=single,
	framesep = 1em,
	numbers=left, 
	breaklines=true, 
	tabsize=4,
	xleftmargin=2em,xrightmargin=2em,
	fontsize=\footnotesize]{xml}
<manifest
	xmlns:android="http://schemas.android.com/apk/res/android"
	xmlns:dist="http://schemas.android.com/apk/distribution"
	package="com.example.myapplication">
	<dist:module dist:instant="true" />
	<application ...>
		...
		<service android:name=".Service1"
			android:enabled="true"
			android:exported="true">
		</service>
		<service
			android:name = ".Service2"
			android:enabled = "true"
			android:exported = "false">
			<intent-filter>
				<category android:name = "cat1"/>
				<action android:name = "act2"/>
			</intent-filter>
		</service>
		<service android:name = ".Service3"
			android:enabled = "true"
			android:permission = "Permission1">
			<intent-filter>
				<action android:name = "act3"/>
				<category android:name = "cat2"/>
				<data android:scheme = "Scheme1"/>
				<data android:scheme = "Scheme2"/>
			</intent-filter>
		</service>
	</application>
</manifest>
	\end{minted}
	\label{declaration:service}
\end{listing}
通常,每个服务都要在\textbf{AndroidManifest.xml}中注册一个\textbf{<service>}标签(参考Listing.\textcolor{red}{\ref{declaration:service}}中的样例)。同时服务可以通过设置“\textbf{android:exported}”属性来指定该服务是否将被导出。若希望服务可以被其他应用启动,则指定\textbf{android:exported = “true”},反之亦然。
%
%
%
%
%可以凑更多字数
%
%
%
%
\subsection{服务的内存泄漏}\label{service_leak}
通常,一个服务的实例不再被使用时应该被\textbf{GC(Garbage Collector)}回收,并释放资源。然而在某些情况下,一个被销毁的服务可能会意外的被引用,从而使得\textbf{GC}无法将其回收并释放资源,这样就造成了服务的内存泄漏。
\begin{listing}[htbp]
	\centering
	\caption{服务的内存泄漏}
	\begin{minted}[encoding=utf8,
	frame=single,
	framesep = 1em,
	numbers=left, 
	breaklines=true, 
	tabsize=4,
	showtabs = false,
	xleftmargin=2em,xrightmargin=2em,
	fontsize=\footnotesize]{java}
public class LeakedService extends Service{
	private static final String TAG = "LeakedService";
	//服务实例的启动方法
	public void onCreate(){
		...
		new Timer().scheduleAtFixedRate(()->{
			Log.d(TAG, LeakedService.this.getPackageName() + ".LeakedService is running!");
		},1000L,3000L);
	}
	//服务实例的销毁方法
	public void onDestroy(){
		...
	}
}
	\end{minted}
	\label{leaked example:service}
\end{listing}

例如在游戏\emph{com.siendas.games2048}中,就出现了原理如图(见\textbf{Listing.\textcolor{red}{\ref{leaked example:service}}})的内存泄漏。具体导致内存泄露的原理为:在\textbf{LeakedService}的实例被构造的时候,将会调用他的\textbf{onCreate()}方法,在该方法中延迟\textbf{1000ms}启动了一个\textbf{匿名}计时器,该计时器将以\textbf{3000ms}的周期打印调试信息,可以看到在\textbf{TimerTask}类中持有了\textbf{LeakedService}的引用,而在该服务被销毁时,其\textbf{onDestroy()}方法中并没有对该匿名计时器进行销毁。因此在该服务被销毁后,将会一直存在一个匿名计时器持有该服务的引用,从而使得\textbf{垃圾回收器(GC)}无法回收该服务资源,从而形成了该服务实例的内存泄漏。

\section{安卓广播接收器}\label{broadcast}

安卓系统中,可以在安卓应用之间、以及安卓应用与安卓系统之间实现通讯,这种通讯的手段被称为广播\cite{androidbroadcastsguide},广播组件的设计采用了发布-订阅的设计模式。在特定的事件发生时,与之相关的广播将会被发送,例如:安卓系统将在各种系统事件(如插拔耳机,插拔电源等)发生时进行对应的系统广播的发送;再如:一个安卓应用可以发送自定义的广播来同志其他的安卓应用,将一些消息传递给后者。

在使用广播时,应用需要注册接收特定的广播(称之为\textbf{订阅订阅})。在相应的广播发出,将由安卓系统负责将广播发送给订阅此种广播的应用。因此在安卓应用的角度,只需要订阅所需要的广播类型,以及关注在接收到广播时需要做出的响应动作即可,具有这种功能的组件被称为\textbf{广播接收器}。
	
一般而言,在跨应用通信时,广播是被广泛使用的一种方便的手段。但是在使用广播时需要格外小心,如果应用滥用广播,可能会导致系统运行变迟钝。更严重的,如果开发人员在编写广播接收器的时候引入\textbf{程序缺陷},将会导致更严重的问题(如应用崩溃等)。

本节接下来将会介绍广播接收器的启动方式,生命周期,注册方式,以及内存泄漏的成因。

\subsection{广播接收器的启动方式}
安卓应用中的广播接收器亦有两种方式启动\cite{broadcast}:

\textbf{清单声明的接收器 }\label{declaration:receiver in manifest} 通过在\textbf{AndroidManifest.xml}中添加\textbf{<receiver>}标签注册广播接收器,通过\textbf{<intent-filter>}标签指定接收器所订阅的广播操作。在应用安装时,将会由软件包管理器注册此类接收器。此后,此广播接收器会变为应用的一种独立的启动方式,也就是说:即使应用未处于运行状态,系统在接到其订阅的广播后,也可启动应用并发送这条广播。在收到特定广播后,系统会创建一个专门的\textbf{BroadcastReceiver}实例,并将广播发送给此实例,由它来完成收到广播后的自定义行为。接收器实例将在\textbf{onReceive(Context, Intent)}方法种获取所订阅的广播内容,并完成相关操作,在从此方法种返回之后,系统便会对接收器进行销毁和资源回收。

\textbf{上下文注册的接收器 }\label{declaration:receiver in context} 通过在代码中构造出\textbf{BroadReceiver}实例,以及\textbf{IntentFilter}实例来规定该接收器所订阅的广播内容,调用\textbf{registerReceiver(BroadcastReceiver, IntentFilter) API}在安卓系统种完成此接收器的注册。通过这种方式注册的接收器的生命周期与注册的上下文有关,只要上下文有效,则接收器可以继续工作,持续接收广播。如果要停止接收广播,需要使用\textbf{unregisterReceiver(BroadcastReceiver) API}来注销此广播接收器,之后系统将会对接收器进行销毁和资源回收,之后不再会接收对应的广播。

\subsection{广播接收器的生命周期}

广播接收器的生命周期根据启动方式不同亦分为两种\cite{broadcast}:

\textbf{清单声明的接收器 } 静态注册的接收器生命周期不限于\textbf{Activity}甚至整个应用。即使应用并不在运行,接收器也可以接收到订阅的广播。将会在\textbf{onReceive()}方法结束后被销毁。


\textbf{上下文注册的接收器 } 上下文注册的接收器,其生命周期仅限于注册的上下文,例如在\textbf{Activity}上下文注册的接收器,在整个\textbf{Activity}存活期间可以持续接收广播;在应用上下文中注册的接收器,则会在整个应用运行期间都可以接收广播。需要注意的是:这种方式启动的接收器必须手动进行销毁,即调用\textbf{unregisterReceiver() API},否则在上下文失效时,系统会抛出异常(并不会导致应用崩溃),同时接收器会引发泄露(见图.\textcolor{red}{\ref{fig:broadcast_leak}})。

\begin{figure}[htbp]
	\centering
	\includegraphics[width=0.9\textwidth]{broadcast_leak.png} % requires the graphicx package
	\caption{没有回收接收器将会导致异常以及泄露}
	\label{fig:broadcast_leak}
\end{figure}

\subsection{广播接收器的注册方式}
\begin{listing}[htbp]
	\centering
	\caption{广播接收器的注册方式}
	\begin{minted}[encoding=utf8,
	frame=single,
	framesep = 1em,
	numbers=left, 
	breaklines=true, 
	tabsize=4,
	showtabs = false,
	xleftmargin=2em,xrightmargin=2em,
	fontsize=\footnotesize]{XML}
<manifest 
	xmlns:android="http://schemas.android.com/apk/res/android"
	xmlns:dist="http://schemas.android.com/apk/distribution"
	package="com.example.myapplication">
	<dist:module dist:instant="true" />
	<application ...>
		...
		<receiver
			android:name = ".Receiver1">
			<intent-filter>
				<action android:name = "act1" />
			</intent-filter>
		</receiver>
		<receiver
			android:name = ".Receiver2"
			android:exported = "false"
			android:enabled = "true">
			<intent-filter>
				<category android:name = "cat1" />
				<action android:name = "act2" />
			</intent-filter>
		</receiver>
	</application>
</manifest>
	\end{minted}
	\label{declaration:receiver}
\end{listing}

一般而言,清单声明的广播接收器(见\ref{declaration:receiver in manifest})需要在\textbf{AndroidManifest.xml}文件中添加\textbf{<receiver>}标签(参考\textbf{Listing.\textcolor{red}{\ref{declaration:receiver}}}),在\textbf{<intent-filter>}子标签中可以指定订阅的广播内容等,也可以通过设置\textbf{“android:exported”}属性来指定该广播是否将被导出。而上下文注册的广播接收器(见\ref{declaration:receiver in context})则不需要进行前文的操作。
\subsection{广播接收器的内存泄漏}
广播接收器的内存泄漏原理类似与服务内存泄漏\ref{service_leak}。但是由广播接收器引起的内存泄漏往往比服务更为严重,因为广播接收器被系统认为只进行不耗时的操作(如果超过10s未从\textbf{onReceive()}方法中返回,将抛出\textbf{ANR(Application Not Response)异常}),因此通常广播接收器在接到广播后,很有可能会启动其他的\textbf{Service}进行后续的耗时操作,进而可能会导致一连串的内存泄漏。

例如图中(见\textbf{Listing.\textcolor{red}{\ref{leaked example:receiver}}})所示的广播接收器,不仅本身会导致内存泄漏,而且还会启动一个会导致内存泄漏的服务(见\textbf{Listing.\textcolor{red}{\ref{leaked example:service}}}),因此后果将会更加严重。
\begin{listing}[htbp]
	\centering
	\caption{广播接收器的内存泄漏}
	\begin{minted}[encoding=utf8,
	frame=single,
	framesep = 1em,
	numbers=left, 
	breaklines=true, 
	tabsize=4,
	showtabs = false,
	xleftmargin=2em,xrightmargin=2em,
	fontsize=\footnotesize]{java}
public class LeakReceiver extends BroadcastReceiver {
	private final String TAG = "LeakReceiver";
	private final int ID = new Random().nextInt();
	@Override
	public void onReceive(Context context, Intent intent) {
		...
		context.startService(new Intent(context,LeakService.class));
		new Timer().scheduleAtFixedRate(()->{
			Log.i(TAG,LeakReceiver.this.ID + " is running!");
		}, 1000L, 3000L);
	}
}
	\end{minted}
	\label{leaked example:receiver}
\end{listing}

\section{电池管理限制}

本节将对电池管理限制进行介绍,同时为了理解电池管理限制的具体生效方式,将对安卓启动服务的源码进行追溯,找到办法绕过电池管理限制。

\subsection{简介}

在安卓9的更新中,引入了\textbf{电池管理}\cite{histroyofpms}来改善安卓设备的电池使用情况,这项更新是先前版本若干相关更新的集成和拓展,目的是帮助节约电量消耗,使得安卓设备的续航得到改善。

其主要限制有两种:

\textbf{应用待机群组} 系统会根据用户使用应用的时间和频率,将应用分成\textbf{活跃}、\textbf{工作集}、\textbf{常用}、\textbf{极少使用}、\textbf{从未使用}五个群组。系统会根据应用所处的群组,来限制应用的行为:例如限制网络连接能力,设定资源分配优先级等。
\newline

\textbf{省电模式改进} 省电模式是在安卓9之前的版本中就已经存在的功能,在安卓9中继续晚上扩展了这一功能:如禁止所有后台应用启动服务和使用网络访问服务等。

\subsection{绕过电池管理限制方法}

首先,启动服务时,需要调用\textbf{startService()}(见图.\redbf{\ref{fig:source code:startService}}) 或者 \textbf{bindService()}(见图.\redbf{\ref{fig:source code:bindService}})  之一API,当调用这两个API时,会进入安卓源码的\textbf{ContextImpl.java}类对应的方法中,可以发现这两个API会分别继续调用\textbf{startServiceCommon()} 和\textbf{bindServiceCommon()} API,于是进一步阅读这两个方法。

\begin{figure}[htbp]
	\centering
	\includegraphics[width=0.9\textwidth]{ContextImpl-startService.png} % requires the graphicx package
	\caption{调用startService() API}
	\label{fig:source code:startService}
\end{figure}

\begin{figure}[htbp]
	\centering
	\includegraphics[width=0.9\textwidth]{ContextImpl-bindService.png} % requires the graphicx package
	\caption{调用bindService() API}
	\label{fig:source code:bindService}
\end{figure}

通过阅读\textbf{startServiceCommon()}(见图.\redbf{\ref{fig:source code:startServiceCommon}}) 和 \textbf{bindServiceCommon()}(见图.\redbf{\ref{fig:source code:bindServiceCommon}})发现了当尝试进行后台启动服务时会抛出的异常信息:\redbf{Not allowed to start service ...},从而确定对后台启动服务的限制具体是由\textbf{AMS}来负责处理,继续阅读\textbf{AMS}代码可以发现,具体的判定逻辑位于\textbf{ActiveServices.java}(见图.\redbf{\ref{fig:source code:getAppStartModeLocked}})以及\textbf{ActivityManagerService.java}(见图.\redbf{\ref{fig:source code:appRestrictedInBackgroundLocked}})两个类中。

\begin{figure}[htbp]
	\centering
	\includegraphics[width=0.9\textwidth]{ContextImpl-startServiceCommon.png} % requires the graphicx package
	\caption{startServiceCommon() API}
	\label{fig:source code:startServiceCommon}
\end{figure}


\begin{figure}[htbp]
	\centering
	\includegraphics[width=0.9\textwidth]{ContextImpl-bindServiceCommon.png} % requires the graphicx package
	\caption{bindServiceCommon() API}
	\label{fig:source code:bindServiceCommon}
\end{figure}

在\textbf{ActiveServices.java}(见图.\redbf{\ref{fig:source code:getAppStartModeLocked}})的\textbf{startServiceLocked()}方法的第\textbf{494}行调用了\textbf{ActivityManagerService.java}(见图.\redbf{\ref{fig:source code:appRestrictedInBackgroundLocked}})的\textbf{getAppStartModeLocked()}方法,在该方法的第\textbf{5955}行的注释发现,安卓会维护一个\textbf{应用白名单},\textbf{在白名单中的应用将会突破“电池管理限制”所限制的后台启动应用限制}。

\begin{figure}[htbp]
	\centering
	\includegraphics[width=0.9\textwidth]{ActiveServices-startServiceLocked.png} % requires the graphicx package
	\caption{getAppStartModeLocked()方法}
	\label{fig:source code:getAppStartModeLocked}
\end{figure}

\begin{figure}[htbp]
	\centering
	\includegraphics[width=0.9\textwidth]{ActivityManagerService-appRestrictedInBackgroundLocked.png} % requires the graphicx package
	\caption{appRestrictedInBackgroundLocked()方法}
	\label{fig:source code:appRestrictedInBackgroundLocked}
\end{figure}

同时经过大量的尝试和探索,在安卓的系统设置中,发现了此白名单的添加方式:\textbf{系统设置 - 应用和通知 - 特殊应用权限 - 电池优化}(见图.\redbf{\ref{fig:hack-battery}})。经过实验,将应用添加至此白名单中,确实可以突破“电池管理限制”,在后台进行服务的启动。在本文后续的测试过程中,所有的应用都将会添加至此白名单中。

\begin{figure}[htbp]
	\centering
	\includegraphics[width=0.9\textwidth]{hack_battery.png} % requires the graphicx package
	\caption{电池优化白名单添加方式}
	\label{fig:hack-battery}
\end{figure}

\section{小结}

本章首先介绍了安卓系统的发展历史,以及安卓应用的四大主要组件,意在说明不可见控件在安卓应用中的重要作用。

接下来重点说明了两类不可见组件(服务和广播)的注册、启动方式,及其生命周期,然后举例说明了组件产生内存泄漏的原因,以及内存泄漏将会导致的严重后果。为后文的检测提供依据。

最后通过追溯安卓源码理解了电池管理的具体限制方式,并通过白名单方式绕过了电池管理限制,使得应用可以从后台启动服务。

\chapter{自动化检测工具}\label{chapter_system}

本章将介绍安卓控件启动的流程,及检测内存泄露的原理。

\section{安卓服务}
安卓应用中的服务可以通过两种方式启动\cite{service}:

\textbf{start 方式 } 其他组件构造特定的\textbf{Intent}对象,通过调用\textbf{startService() API}来启动目标服务。

\textbf{bind 方式 } 通过调用\textbf{bindService() API}将目标服务与特定组件绑定。被绑定的服务提供接口供其他组件与之交互。
一个服务可以同时通过以上两种方式启动。

\subsection{服务的生命周期}
服务的生命周期根据启动方式不同分为两种\cite{service}:

\textbf{start 方式 } 通过\textbf{startService() API}启动的服务将会一直运行,直到调用\textbf{stopSelf()}方法将自己停止运行。其他组件也可以通过调用\textbf{stopService() API}将服务停止运行。

停止运行的服务将会被\textbf{GC(Garbage Collector)}回收。

\textbf{bind 方式 } 通过\textbf{bindService() API}启动的服务将通过\textbf{IBinder}接口与其他组件进行交互,直到其他组件调用\textbf{unbindService() API}解除绑定。

一个服务可以同时绑定到多个组件之上,直到所有组件都解除了绑定时,该服务才会被\textbf{GC}回收。

每个安卓应用都关联一个\textbf{ActivityThread}实例,负责调度和执行该应用的各种组件。\textbf{ActivityThread}有一个\textbf{ArrayMap}类型的成员变量\textbf{mServices},其中保存了所有没有被销毁的服务的引用。一旦某个服务的实例被销毁,其引用将会从\textbf{mServices}中删除。

\subsection{服务的注册方式}
\begin{listing}[htbp]
	\centering
	\caption{服务的注册方式}
	\begin{minted}[encoding=utf8,
	frame=single,
	framesep = 1em,
	numbers=left, 
	breaklines=true, 
	tabsize=4,
	xleftmargin=2em,xrightmargin=2em,
	fontsize=\footnotesize]{xml}
<manifest
    xmlns:android="http://schemas.android.com/apk/res/android"
    xmlns:dist="http://schemas.android.com/apk/distribution"
    package="com.example.myapplication">
    <dist:module dist:instant="true" />
    <application ...>
        ...
        <service android:name=".Service1"
            android:enabled="true"
            android:exported="true">
        </service>
        <service
            android:name = ".Service2"
            android:enabled = "true"
            android:exported = "false">
            <intent-filter>
                <category android:name = "cat1"/>
                <action android:name = "act2"/>
            </intent-filter>
        </service>
        <service android:name = ".Service3"
            android:enabled = "true"
            android:permission = "Permission1">
            <intent-filter>
                <action android:name = "act3"/>
                <category android:name = "cat2"/>
                <data android:scheme = "Scheme1"/>
                <data android:scheme = "Scheme2"/>
            </intent-filter>
        </service>
    </application>
</manifest>
	\end{minted}
	\label{declaration:service}
\end{listing}
通常,每个服务都要在\textbf{AndroidManifest.xml}中注册一个\textbf{<service>}标签(参考Listing.\textcolor{red}{\ref{declaration:service}}中的样例)。同时服务可以通过设置"\textbf{android:exported}"属性来指定该服务是否将被导出。当设置\textbf{android:exported = "true"}时,该服务可以被其他应用使用,反之不可。
%可以凑更多字数
\subsection{服务的内存泄漏}\label{service_leak}
通常,一个服务的实例不再被使用时应该被\textbf{GC(Garbage Collector)}回收,并释放资源。然而在某些情况下,一个被销毁的服务可能会意外的被引用,从而使得\textbf{GC}无法将其回收并释放资源,这样就造成了服务的内存泄漏。
\begin{listing}[htbp]
	\centering
	\caption{服务的内存泄漏}
	\begin{minted}[encoding=utf8,
		frame=single,
		framesep = 1em,
		numbers=left, 
		breaklines=true, 
		tabsize=4,
		showtabs = false,
		xleftmargin=2em,xrightmargin=2em,
		fontsize=\footnotesize]{java}
public class LeakedService extends Service{
	private static final String TAG = "LeakedService";
	// Method will be called when an instance is creating.
	public void onCreate(){
		...
		new Timer().scheduleAtFixedRate(new TimerTask(){
			public void run(){
				Log.d(TAG, LeakedService.this.getPackageName() + ".LeakedService is running!");
			}
		},1000L,3000L);
	}
	// Method will be called when an instance is destroying.
	public void onDestroy(){
		...
	}
}
	\end{minted}
	\label{leaked example:service}
\end{listing}

例如在游戏\emph{com.siendas.games2048}中,就出现了原理如图(见\textbf{Listing.\textcolor{red}{\ref{leaked example:service}}})的内存泄漏。具体导致内存泄露的原理为:在\textbf{LeakedService}的实例被构造的时候,将会调用他的\textbf{onCreate()}方法,在该方法中延迟\textbf{1000ms}启动了一个\textbf{匿名}计时器,该计时器将以\textbf{3000ms}的周期打印调试信息,可以看到在\textbf{TimerTask}类中持有了\textbf{LeakedService}的引用,而在该服务被销毁时,其\textbf{onDestroy()}方法中并没有对该匿名计时器进行销毁。因此在该服务被销毁后,将会一直存在一个匿名计时器持有该服务的引用,导致\textbf{GC}无法将其回收,从而导致了内存泄漏。

\section{安卓广播接收器}
安卓应用中的广播接收器亦有两种方式启动\cite{broadcast}:

\textbf{清单声明的接收器 }\label{declaration:receiver in manifest} 通过在\textbf{AndroidManifest.xml}中添加\textbf{<receiver>}标签注册广播接收器,通过\textbf{<intent-filter>}标签指定接收器所订阅的广播操作。系统软件包管理器会在应用安装时注册接收器。然后,该接收器会成为应用的一个独立入口点,这意味着如果应用当前尚未运行,系统可以启动应用并发送广播。系统会创建新的\textbf{BroadcastReceiver}组件对象来处理它接收到的每个广播。次对象仅在调用\textbf{onReceive(Context, Intent)}期间有效。一旦从此方法返回代码,系统便会认为该组件不再活跃。

\textbf{上下文注册的接收器 }\label{declaration:receiver in context} 通过在代码中构造出\textbf{BroadReceiver}实例,以及\textbf{IntentFilter}实例来指定订阅的广播内容,调用\textbf{registerReceiver(BroadcastReceiver, IntentFilter) API}来注册接收器。只要上下文有效,通过改种方式注册的广播接收器就会接收广播。如果要停止接收广播,需要调用\textbf{unregisterReceiver(BroadcastReceiver) API}来注销广播接收器

\subsection{广播接收器的生命周期}
广播接收器的生命周期根据启动方式不同亦分为两种\cite{broadcast}:

\textbf{清单声明的接收器 } 静态注册的接收器生命周期不限于\textbf{Activity}甚至整个应用。即使应用并不在运行,接收器也可以接收到订阅的广播。将会在\textbf{onReceive()}方法结束后被销毁。


\textbf{上下文注册的接收器 } 上下文注册的接收器,其生命周期仅限于注册的上下文,例如在\textbf{Activity}上下文注册的接收器,在整个\textbf{Activity}存活期间可以持续接收广播;在应用上下文中注册的接收器,则会在整个应用运行期间都可以接收广播。需要注意的是:这种方式启动的接收器必须手动进行销毁,即调用\textbf{unregisterReceiver() API},否则在上下文失效时,系统会抛出异常(并不会导致应用崩溃),同时接收器会引发泄露(见图.\textcolor{red}{\ref{fig:broadcast_leak}})。

\begin{figure}[htbp]
   \centering
   \includegraphics[width=0.9\textwidth]{broadcast_leak.png} % requires the graphicx package
   \caption{没有回收接收器将会导致异常以及泄露}
   \label{fig:broadcast_leak}
\end{figure}

\subsection{广播接收器的注册方式}
\begin{listing}[htbp]
	\centering
	\caption{广播接收器的注册方式}
	\begin{minted}[encoding=utf8,
	frame=single,
	framesep = 1em,
	numbers=left, 
	breaklines=true, 
	tabsize=4,
	showtabs = false,
	xleftmargin=2em,xrightmargin=2em,
	fontsize=\footnotesize]{XML}
<manifest 
	xmlns:android="http://schemas.android.com/apk/res/android"
	xmlns:dist="http://schemas.android.com/apk/distribution"
	package="com.example.myapplication">
	<dist:module dist:instant="true" />
	<application ...>
		...
		<receiver
			android:name = ".Receiver1">
			<intent-filter>
				<action android:name = "act1" />
			</intent-filter>
		</receiver>
		<receiver
			android:name = ".Receiver2"
			android:exported = "false"
			android:enabled = "true">
			<intent-filter>
				<category android:name = "cat1" />
				<action android:name = "act2" />
			</intent-filter>
		</receiver>
	</application>
</manifest>
	\end{minted}
	\label{declaration:receiver}
\end{listing}

一般而言,清单声明的广播接收器(见\ref{declaration:receiver in manifest})需要在\textbf{AndroidManifest.xml}文件中添加\textbf{<receiver>}标签(参考\textbf{Listing.\textcolor{red}{\ref{declaration:receiver}}}),在\textbf{<intent-filter>}子标签中可以指定订阅的广播内容等,也可以通过设置\textbf{"android:exported"}属性来指定该广播是否将被导出。而上下文注册的广播接收器(见\ref{declaration:receiver in context})则不需要进行前文的操作。
\subsection{广播接收器的内存泄漏}
广播接收器的内存泄漏原理类似与服务内存泄漏\ref{service_leak}。但是由广播接收器引起的内存泄漏往往比服务更为严重,因为广播接收器被系统认为只进行不耗时的操作(如果超过10s未从\textbf{onReceive()}方法中返回,将抛出\textbf{ANR Exception}),因此通常广播接收器在接到广播后,很有可能会启动其他的\textbf{Service}进行后续的耗时操作,进而可能会导致一连串的内存泄漏。

例如图中(见\textbf{Listing.\textcolor{red}{\ref{leaked example:receiver}}})所示的广播接收器,不仅本身会导致内存泄漏,而且还会启动一个会导致内存泄漏的服务(见\textbf{Listing.\textcolor{red}{\ref{leaked example:service}}}),因此后果将会更加严重。
\begin{listing}[htbp]
	\centering
	\caption{广播接收器的内存泄漏}
	\begin{minted}[encoding=utf8,
	frame=single,
	framesep = 1em,
	numbers=left, 
	breaklines=true, 
	tabsize=4,
	showtabs = false,
	xleftmargin=2em,xrightmargin=2em,
	fontsize=\footnotesize]{java}
public class LeakReceiver extends BroadcastReceiver {
	private final String TAG = "LeakReceiver";
	private final int ID = new Random().nextInt();
	@Override
	public void onReceive(Context context, Intent intent) {
		...
		context.startService(new Intent(context,LeakService.class));
		new Timer().scheduleAtFixedRate(new TimerTask() {
			@Override
			public void run() {
				Log.i(TAG,LeakReceiver.this.ID + " is running!");
			}
		}, 1000L, 3000L);
	}
}
	\end{minted}
	\label{leaked example:receiver}
\end{listing}
\section{自动化检测工具原理}
\begin{figure}[htbp]
	\centering
	\includegraphics[width=0.9\textwidth]{main_flow.png} % requires the graphicx package
	\caption{自动检测工具原理}
	\label{fig:flow}
\end{figure}

如图\textbf{\textcolor{red}{\ref{fig:flow}}}所示,测试将在两个环境(工作站、安卓模拟器)中完成:
\begin{enumerate}
	\item 在工作站中,使用\textbf{Apk文件解析器}(详见\textbf{\textcolor{red}{\ref{apk analyser}}})对被测试应用\textbf{Test.apk}进行反编译,并解析\textbf{AndroidManifest.xml}清单,得到在清单中注册的\textbf{公开服务}以及\textbf{清单声明的广播接收器},作为测试对象生成一份\textbf{测试配置}发送到安卓模拟器上。
	\item 在安卓模拟器上安装好\textbf{测试驱动App}(详见\redbf{\ref{test driver app}})以及被测试应用\textbf{Test.apk}。通过\textbf{ADB}向安卓模拟器发送指令启动\textbf{测试驱动App},它会读取步骤1中发送来的\textbf{测试配置},然后执行测试主流程(详见\redbf{\ref{main flow}})
	\item 在完成测试主流程之后,通知工作站将被测试应用的堆镜像文件(.prof)文件导出,使用\textbf{内存分析器}(详见\redbf{\ref{memory analyser}})进行内存泄漏的检测,最终生成一份\textbf{内存泄漏报告}显示被测试应用的服务以及广播接收器有无内存泄漏情况,以及具体的内存泄漏的控件清单。
\end{enumerate}

\subsection{Apk 文件解析器}\label{apk analyser}
Apk 文件解析器旨在通过将被测试应用反编译得到\textbf{AndroidManifest.xml}清单,对该清单进行分析,从而得到在清单中声明的\textbf{公开服务}以及\textbf{广播接收器}作为测试对象。
\begin{listing}[htbp]
	\centering
	\caption{使用apktool工具进行apk的反编译}
	\begin{minted}[encoding=utf8,
	frame=single,
	framesep = 1em,
	numbers=left, 
	breaklines=true, 
	tabsize=4,
	showtabs = false,
	xleftmargin=2em,xrightmargin=2em,
	fontsize=\footnotesize]{shell}
apktool d -f $test_apk$ -o temp
	\end{minted}
	\label{shell:apktool}
\end{listing}

使用apktool工具\cite{apktool},执行\textbf{Listing.}\redbf{\ref{shell:apktool}}中的命令即可完成被测试应用的反编译得到\textbf{AndroidManifest.xml}清单。

\begin{listing}[htbp]
	\centering
	\caption{使用python解析xml输出测试配置}
	\begin{minted}[encoding=utf8,
	frame=single,
	framesep = 1em,
	numbers=left, 
	breaklines=true, 
	tabsize=4,
	showtabs = false,
	xleftmargin=2em,xrightmargin=2em,
	fontsize=\footnotesize]{python}
from xml.dom.minidom import parse
def get_exported_services(xml_path):
	exported_services = []
	manifest = parse(xml_path).documentElement 
	for service in manifest.getElementsByTagName('service'):
		exported = service.getAttribute('android:exported')
		enabled = service.getAttribute('android:enabled')
		if (exported == 'true' and (enabled == '' or enabled == 'true')):
			exported_services.append(service)
	return exported_services
	
def get_static_receivers(xml_path):
	exported_receivers = []
	domTree = parse(xml_path)
	manifest = domTree.documentElement
	for receiver in manifest.getElementsByTagName('receiver'):
		enabled = receiver.getAttribute('android:enabled')
		if (enabled == '' or enabled == 'true'):
			exported_receivers.append(receiver)
	return exported_receivers;
	\end{minted}
	\label{python:get services}	
\end{listing}

接下来使用\textbf{python}中的\textbf{xml} 库解析此xml文件(详见\textbf{Listing.}\redbf{\ref{python:get services}}),筛选出所有指定\textbf{android:exported = "true"} 以及 \textbf{android:enabled = "true"}(或者不指定,默认值为"true")的服务,以及在清单中声明的广播接收器。将这些控件的包名,类名,以及所需的权限等信息写入\textbf{测试配置}中
\subsection{测试驱动App}\label{test driver app}
测试驱动App负责控制所有安卓模拟器上的测试行为(即测试主流程\redbf{\ref{main flow}})。首先驱动会读取\textbf{测试配置},获得要测试的服务和广播接收器,之后进行\textbf{测试主流程},在完成测试流程之后,使用\textbf{Socket}与\textbf{工作站}进行通信,通知\textbf{工作站}测试任务已经完成,导出被测试应用的堆镜像文件进行最后的分析。
\subsection{测试主流程}\label{main flow}

\begin{algorithm}
	\caption{测试主流程:公开服务测试}
	\label{alg:service}
	\begin{algorithmic}[1]
		\STATE {读取测试配置}
		\FOR{遍历测试配置中的公开服务}
			\STATE{重置安卓模拟器}
			\STATE{重置计数器}
			\IF{使用bind方式测试}
				\STATE{启动仿真Activity}
			\ENDIF
			\WHILE{计时器 < 测试重复次数}
				\IF{使用start方式测试}
					\STATE {调用startService() API 启动服务} 
					\STATE {调用stopService() API 停止服务}
				\ELSE
					\STATE{调用bindService() API 将服务绑定到Activity上}
					\STATE{调用unbindService() API 解除服务绑定}
				\ENDIF
				\STATE{计数器自增1}
			\ENDWHILE
			\IF{使用bind方式测试}
				\STATE{销毁仿真Activity}
			\ENDIF
			\STATE{通知工作站导出堆镜像}
		\ENDFOR
	\end{algorithmic}
\end{algorithm}

\begin{algorithm}
	\caption{测试主流程:清单声明的广播接收器}
	\label{alg:receiver}
	\begin{algorithmic}[1]
		\STATE {读取测试配置}
		\FOR{遍历测试配置中的广播接收器}
			\STATE{重置安卓模拟器}
			\STATE{重置计数器}
			\WHILE{计时器 < 测试重复次数}
				\STATE{构造特定广播,定向发送给该接收器}
				\STATE{计数器自增1}
			\ENDWHILE
			\STATE{通知工作站导出堆镜像}
		\ENDFOR
	\end{algorithmic}
\end{algorithm}

在测试主流程中,对服务和接收器进行逐一测试:

\textbf{服务测试(见算法\redbf{\ref{alg:service}}) }  测试原理为对被测试应用进行大量重复的启动和关闭(根据测试要求不同分为start/bind 和stop/unbind),确保潜在的内存泄漏被触发,之后将被测试应用的堆镜像导出,送入\textbf{内存分析器}(详见\redbf{\ref{memory analyser}})进行检测和分析。

\textbf{广播接收器测试(见算法\redbf{\ref{alg:receiver}}) }
广播接收器的测试不需要事先启动应用,因为在清单中注册的广播接收器在应用安装时就会在系统中进行注册,在任何时候都可以接收广播(即使应用未运行),因此只需要构造接收器订阅的广播,定向发送给该接收器就可以触发该接收器。同样的为了确保潜在的内存泄漏被触发,需要重复发送大量的广播。最终将被测试应用的堆镜像导出,送入\textbf{内存分析器}(详见\redbf{\ref{memory analyser}})进行检测和分析。
\subsection{内存分析器}\label{memory analyser}

\begin{algorithm}
	\caption{内存分析器:服务分析}
	\label{alg:memory analyser:service}
	\begin{algorithmic}[1]
		\STATE{导入并解析.hprof文件}
		\STATE{查询所有Service的衍生类集合}
		\FOR {遍历Service的衍生类集合}
			\IF{该实例已经被销毁}
				\STATE{获得所有该实例到GC Root的路径集合}
				\STATE{去除路径集合中不合理的路径}
				\IF[证明该实例确实已经被销毁]{路径集合为空}
					\STATE{该实例不存在内存泄漏}
				\ELSE[该实例仍然被引用,产生内存泄漏]
					\STATE{该实例产生内存泄漏}
					\IF{该实例内存泄漏原因为已知的安卓系统BUG}
						\STATE{不判定构成人为内存泄漏}
					\ELSE
						\STATE{由开发者构成的人为内存泄漏}
					\ENDIF
				\ENDIF
			\ENDIF
		\ENDFOR
	\end{algorithmic}
\end{algorithm}

\begin{algorithm}
	\caption{内存分析器:服务分析}
	\label{alg:memory analyser:receiver}
	\begin{algorithmic}[1]
		\STATE{导入并解析.hprof文件}
		\STATE{查询所有BroadcastReciver的衍生类集合}
		\FOR {遍历BroadcastReceiver的衍生类集合}
			\IF{该实例已经被销毁}
				\STATE{获得所有该实例到GC Root的路径集合}
				\STATE{去除路径集合中不合理的路径}
				\IF[证明该实例确实已经被销毁]{路径集合为空}
					\STATE{该实例不存在内存泄漏}
				\ELSE[该实例仍然被引用,产生内存泄漏]
					\STATE{该实例产生内存泄漏}
					\IF{该实例内存泄漏原因为已知的安卓系统BUG}
						\STATE{不判定构成人为内存泄漏}
					\ELSE
						\STATE{由开发者构成的人为内存泄漏}
					\ENDIF
				\ENDIF
			\ENDIF
		\ENDFOR
	\end{algorithmic}
\end{algorithm}

内存分析器负责从堆镜像文件(\textbf{.hprof文件})中识别和统计服务(广播接收器)的内存泄漏实例,基于开源工具MAT\cite{mat}进行定制化的开发,具体原理(详见\textbf{算法\redbf{\ref{alg:memory analyser:service}}及\redbf{\ref{alg:memory analyser:receiver}}})为:首先导入并解析堆镜像文件,找到所有的服务(广播接收器)实例,接下来对于每个找到的实例,逐个检测判断是否产生了内存泄露,对于产生内存泄漏的控件,生成一份内存泄漏分析报告。

\textbf{服务的内存泄露判定 } 所有仍然活跃的服务都会被\textbf{ActivityThread} 的 \textbf{mServices}变量所引用,而所有被销毁的服务都会被从\textbf{mServices}中删除。一个对象的实例如果实际上处于存活状态,则一定会拥有至少一条有效的\textbf{GC Root Path}。因此服务内存泄漏的判定充要条件为:拥有至少一条有效的\textbf{GC Root Path}且不在\textbf{mServices}中。

\textbf{广播接收器的内存泄漏判定 } 由于广播接收器被系统规定为不耗时控件(超过10s未完成\textbf{onReceive()} 方法,将会抛出\textbf{ANR Exception}导致应用崩溃),因此对于广播接收器而言,只需要检测是否有实际处于存活状态的实例即可。即:拥有至少一条有效的\textbf{GC Root Path}。

\textbf{安卓系统自身原因产生内存泄漏 } 由于安卓系统自身可能存在\textbf{BUG}导致正常的控件产生内存泄漏,此类问题并非由于安卓应用开发者的失误而导致,因此要将此类内存泄漏排除。

\subsection{实现细节}

\textbf{后台服务限制 }\cite{android-service-limit} 在安卓8中,新增了“电池优化策略”,系统不允许后台应用创建后台服务,因此跨应用对服务进行测试将会失败。解决办法为:禁用系统的电池管理服务;在测试时,将被测试应用置于前台。

\textbf{广播限制 }\cite{android-receiver-limit} 在\textbf{安卓8}及更高版本系统的应用中禁止将隐式广播注册为清单声明的广播接收器。而显示广播和需要签名授权的广播接收器不受限制,可以继续注册为清单声明的广播接收器。\textbf{注意:}随着安卓系统的更新,很多隐式广播已经不再受到此规定的限制,具体的广播列表可以参考隐式广播例外\cite{android-receiver-limit-exception}。

\textbf{超级用户限制 } 在\textbf{安卓7}以及更早的版本中,开发者经常会将系统增加超级用户权限,以便于进行测试(亦成为\textbf{root}),大致的方法为将兼容的二进制文件\textbf{su}拷贝到安卓设备中,使安卓设备可以执行超级用户指令\textbf{sudo},进而为测试带来便利。然而再\textbf{安卓8}以及更高的系统中,由于用于\textbf{root}操作的二进制文件\textbf{su}维护团队相继解散,获得非安卓系统开发人员获得超级用户权限变得困难,因此本文的自动化检测工具的实现不要求对系统进行\textbf{root},而将使用\textbf{socket}建立安卓设备与工作站的通信,由工作站使用\textbf{adb}指令完成跨应用的特殊权限操作:比如\textbf{跨应用导出堆镜像文件}操作需要应用拥有超级用户权限,而使用\textbf{adb}指令则不需要超级用户权限即可导出任意应用的堆镜像文件。

\chapter{实验}\label{chapter_experiment}

在实验中,首先使用自动化检测工具对仿真应用进行了测试,以验证可行性和正确性。接下来对真实的应用进行了测试,收集了测试数据。

\section{仿真测试}

\subsection{仿真应用}

仿真应用(见\textbf{Listing.}\redbf{\ref{code: manifest}})应同时具有两个\textbf{公开服务}以及两个\textbf{清单声明的广播接收器}。其中每个控件之一需要人为制造内存泄漏(称为\textbf{LeakedService(Listing.\redbf{\ref{code:LeakedService}})}与\textbf{LeakedReceiver(Listing.\redbf{\ref{code:LeakedReceiver}})}),另一个则需要确保不存在内存泄漏(称为\textbf{NormalService(Listing.\redbf{\ref{code:Normal}})}与\textbf{NormalReceiver(Listing.\redbf{\ref{code:Normal}})})。
在对仿真应用进行测试时,预期的实验结果为:能够检测到\textbf{LeakedService}和\textbf{LeakedReceiver}的泄露实例,以证明该工具可以发现内存泄漏问题。而检测不到\textbf{NormalService}和\textbf{NormalReceiver}的泄露实例,以证明该工具不会将正常的组件误检。


\begin{listing}[htbp]
	\centering
	\caption{\textbf{LeakedService}主体代码}
	\begin{minted}[encoding=utf8,
	frame=single,
	framesep = 1em,
	numbers=left, 
	breaklines=true, 
	tabsize=4,
	xleftmargin=2em,xrightmargin=2em,
	fontsize=\footnotesize]{java}
public class LeakedService extends Service {
	private static final String TAG = "LeakedService";
	@Override
	public void onCreate() {
		super.onCreate();
		new Timer().scheduleAtFixedRate(new TimerTask() {
			@Override
			public void run() {
				Log.i(TAG,LeakService.this.getPackageName() + ".LeakService running ");
			}
		},1000L,3000L);
	}
}	
	\end{minted}
	\label{code:LeakedService}
\end{listing}

\begin{listing}[htbp]
	\centering
	\caption{\textbf{LeakedReceiver}主体代码}
	\begin{minted}[encoding=utf8,
	frame=single,
	framesep = 1em,
	numbers=left, 
	breaklines=true, 
	tabsize=4,
	xleftmargin=2em,xrightmargin=2em,
	fontsize=\footnotesize]{java}
public class LeakedReceiver extends BroadcastReceiver {
	private static final String TAG = "LeakedReceiver";
	private final Random random = new Random();
	@Override
	public void onReceive(Context context, Intent intent) {
		new Timer().scheduleAtFixedRate(new TimerTask() {
			@Override
			public void run() {
				Log.i(TAG,LeakReceiver.this.random.nextInt() + ".LeakReceiver running ");
			}
		},1000L,3000L);
	}
}
	\end{minted}
	\label{code:LeakedReceiver}
\end{listing}

\begin{listing}[htbp]
	\centering
	\caption{\textbf{NormalReceiver}与\textbf{NormalService}主体代码}
	\begin{minted}[encoding=utf8,
	frame=single,
	framesep = 1em,
	numbers=left, 
	breaklines=true, 
	tabsize=4,
	xleftmargin=2em,xrightmargin=2em,
	fontsize=\footnotesize]{java}
public class NormalReceiver extends BroadcastReceiver {
	private static final String TAG = "NormalReceiver";
	private final Random random = new Random();
	@Override
	public void onReceive(Context context, Intent intent) {
		Log.i(TAG,NormalReceiver.this.random.nextInt() + ".NormalReceiver running ");
		}
	}
}

public class NormalService extends Service{
	private static final String TAG = "NormalService";
		@Override
	public void onReceive(Context context, Intent intent) {
		Log.i(TAG,NormalService.this.getPackageName() + ".LeakService running ");
	}
}
	\end{minted}
	\label{code:Normal}
\end{listing}

\begin{listing}[htbp]
	\centering
	\caption{仿真应用的AndroidManifest.xml清单}
	\begin{minted}[encoding=utf8,
	frame=single,
	framesep = 1em,
	numbers=left, 
	breaklines=true, 
	tabsize=4,
	xleftmargin=2em,xrightmargin=2em,
	fontsize=\footnotesize]{xml}
<manifest
	xmlns:android="http://schemas.android.com/apk/res/android"
	xmlns:dist="http://schemas.android.com/apk/distribution"
	package="com.example.myapplication">
	<dist:module dist:instant="true" />

	<permission
		android:name="app.custom.permission"
		android:protectionLevel="signature" />
	<application ...>
		<activity android:name=".MainActivity">
			<intent-filter>
				<action android:name="android.intent.action.MAIN" />
				<category android:name="android.intent.category.LAUNCHER" />
			</intent-filter>
		</activity>
		
		<receiver
			android:name=".LeakedReceiver"
			android:exported="true">
			<intent-filter>
				<action android:name="TestActionForLeaked" />
			</intent-filter>
		</receiver>
		
		<receiver
			android:name = ".NormalReceiver"
			android:exported = "true">
			<intent-filter>
				<action android:name = "TestActionForNormal"/>
			</intent-filter>
		</receiver>
		
		<service
			android:name=".LeakedService"
			android:exported="true">
		</service>
	
		<service
			android:name = ".NormalService"
			android:exported = "true">
		</service>
	</application>
</manifest>
	\end{minted}
	\label{code: manifest}
\end{listing}

\subsection{实验结果}
实验结果(见\textbf{图.}\redbf{\ref{fig:result of mock receiver}}及\textbf{图.}\redbf{\ref{fig:result of mock service}})能正确检测到\textbf{LeakedService}和\textbf{LeakedReceiver}的内存泄漏实例,而没有误检\textbf{NormalService}以及\textbf{NormalReceiver},证明检测工具实际有效。

\begin{figure}[htbp]
	\centering
	\includegraphics[width=0.9\textwidth]{service_leak_result.png} % requires the graphicx package
	\caption{检测到\textbf{LeakedService}内存泄漏实例}
	\label{fig:result of mock service}
\end{figure}
\begin{figure}[htbp]
\centering
\includegraphics[width=0.9\textwidth]{receiver_leak_result.png} % requires the graphicx package
\caption{检测到\textbf{LeakedReceiver}内存泄漏实例}
\label{fig:result of mock receiver}
\end{figure}

\section{真实测试}


本文选取了\textbf{AppChina 应用市场}\cite{appchina}作为测试应用的来源,在其中选取了各种类别应用中下载量最高的若干应用进行测试。对\textbf{22}个实际应用的测试结果如下:

\subsection{实验结果}\label{now-result}
\textbf{表现正常 12(54.5\%) }这些应用的组件表现正常,并没有出现内存泄露问题。

\textbf{内存泄漏 3(13.6\%) }这些应用包含了存在内存泄漏问题的组件,并成功被检测工具检测到。

\textbf{应用崩溃 7(31.8\%) }这些应用在测试时抛出了\textbf{ANR(Application Not Response)}异常,导致应用崩溃,无法完成测试。

\subsection{应用崩溃主要原因分析}

\textbf{无法正常启动 } 部分应用在启动时即发生了崩溃,导致应用停止运行。这类应用崩溃的原因可能是因为应用的版本和模拟器系统的版本不兼容,例如使用了不再符合开发规范的接口,某些接口不再支持,或没有在\textbf{AndroidManifest.xml}中声明所需的权限(缺少\textbf{<uses-permission>}标签)等。

\textbf{在测试过程中崩溃 } 部分应用可以正常启动,但是在\textbf{主测试流程}中发生了应用崩溃问题。这类应用崩溃的原因可能是因为应用的组件启动流程存在问题,比如:进行了风险操作,进行了线程不安全操作,没有对启动环境进行检查等;也可能是由于应用的组件确实存在\textbf{内存泄漏}问题,而且泄露表现的十分严重,触发了安卓系统的安全限制,导致安卓系统介入将应用停止运行。

\textbf{空指针异常 } 部分应用的组件在启动时会抛出\textbf{NullPointerException},该异常表示,在组件的启动过程中,对未实例化的对象进行了读写操作。这类问题大多数是因为组件的开发人员的疏忽,导致的程序缺陷。可能的成因有组件之间使用了共享资源,但是并没有进行专门管理,也有可能是因为这类组件的启动严格遵循\textbf{状态自动机},但是测试时无法得知组件启动需要满足的前置条件,导致运行出错。

\section{结果分析}

本小节会将本文的测试结果(见\ref{now-result})与之前的结果\cite{jun2018lesdroid}进行对比,进行分析。

\subsection{数据对比}
\begin{table}[htb]\footnotesize
	\centering
	\caption{不同安卓版本上实验数据对比}
	\vspace{2mm}
	% l - left, r - right, c - center. | means one vertical line 这里声明的是表格单元中的内容如何对齐
	\begin{tabular}{lcccc}
		\toprule
		&\textbf{表现正常}&\textbf{内存泄漏}&\textbf{运行异常}&\textbf{泄露正常比}\\
		\midrule
		\textbf{安卓6}&63.8\%&13.7\%&22.5\%&0.215\\
		\hline
		\textbf{安卓9}&54.5\%&13.6\%&31.8\%&0.250\\
		\bottomrule
	\end{tabular}
	\label{table:compare}
\end{table}

实验数据的对比表明:
\begin{itemize}
	\item 存在内存泄漏的应用比例大致相同,但是表现正常的应用比例有下降,相应的运行异常的应用比例增加。
	\item 泄露正常比(检测到内存泄漏问题的应用数量与表现正常的应用数量之比)上升,这表明在\textbf{稳定可用的安卓应用}中,内存泄漏问题比早先的版本变得更多了。
	\item 运行异常的应用比例增加,原因可能是安卓系统版本升级后,将系统权限收紧,导致在旧的安卓版本上可以获取相应权限正常运行的应用,在新版本的安卓系统上无法正常获取到系统权限(可能是系统权限不再可以获取,或者获取权限方式改变等),从而导致应用权限不足,无法正常运行;另外的可能原因是安卓版本升级之后,新增或变更了部分开发规范,导致原有应用的代码在新版本安卓系统上不在兼容(例如新版本不在允许启动后台服务,而需要启动前台服务,并显式获得用户许可等)。
\end{itemize}

\subsection{实验数据局限性}

由于实验设备和实验环境的限制(参考\textbf{表.}\redbf{\ref{table:pc-compare}}),本文只能使用效率较慢的方式对少量小规模的应用进行串行的测试(参考\textbf{表.}\redbf{\ref{table:method-compare}})。因此本文得到的实验数据结果具有局限性,说服力较小。

\begin{table}[htb]\footnotesize
	\centering
	\caption{工作站配置对比}
	\vspace{2mm}
	% l - left, r - right, c - center. | means one vertical line 这里声明的是表格单元中的内容如何对齐
	\begin{tabular}{lcccc}
		\toprule
		&\textbf{操作系统}&\textbf{内存容量}&\textbf{处理器型号}&\textbf{模拟器系统版本}\\
		\midrule
		\textbf{LESDroid测试配置\cite{jun2018lesdroid}}&Windows 10&32 GB&Intel Xeon E5-1650&Android 6\\
		\hline
		\textbf{本文检测工具的测试配置}&Windows 10&8 GB&Intel(R) Core(TM) i7-8565U&Android 9\\
		\bottomrule
	\end{tabular}
	\label{table:pc-compare}
\end{table}

\begin{table}[htb]\footnotesize
	\centering
	\caption{测试方法对比}
	\vspace{2mm}
	% l - left, r - right, c - center. | means one vertical line 这里声明的是表格单元中的内容如何对齐
	\begin{tabular}{lcccc}
		\toprule
		&\textbf{并发能力}&\textbf{模拟器内存}&\textbf{模拟器SD卡容量}&\textbf{测试强度}\\
		\midrule
		\textbf{LESDroid测试配置\cite{jun2018lesdroid}}&5线程并发&2 GB&2 GB&每个应用测试3次\\
		\hline
		\textbf{本文检测工具的测试配置}&单线程&2 GB&512 MB&每个应用测试2次\\
		\bottomrule
	\end{tabular}
	\label{table:method-compare}
\end{table}

\chapter{结论}\label{chapter_conclusion}
本文实现了一个自动化的检测工具,它可以帮助安卓应用的开发人员对应用进行测试,帮助发现存在内存泄露问题的公开服务以及清单声明的广播接收器,以便于给用户带来更好的体验。

本自动化检测工具首先对被测试应用进行反编译,分析得到公开服务和清单声明的广播接收器等被测试组件信息;启动安卓模拟器,将被测试应用以及测试驱动应用安装到模拟器上进行测试流程;接下来对被测试组件进行反复的启动/关闭,绑定/解绑定,注册/解注册操作,最大程度确保内存泄漏问题被实际触发;最终为了找到内存泄漏的证据,将被测试应用的堆镜像文件导出到工作站中,基于开源工具开发了检测安卓组件内存泄漏的工具,最终导出内存泄漏报告。

安卓应用的开发人员只需要运行本文中的工具对其开发的应用进行检测,然后阅读内存泄漏报告,就可以对应用进行安全性的评估。

\bibliography{sample}

%%%%%%%%%%%%%%%%%%%%%%%%%%%%%%%%%%%%%%%%%%%%%%%%%%%%%%%%%%%%%%%%%%%%%%%%%%%%%%%
% 致谢,应放在结论之后
\begin{acknowledgement}
由衷地感谢我的导师马骏副教授对我进行了悉心的指导,恩施思维敏捷,脚踏实地,对我实验过程中遇到的苦难和疑惑,都进行了仔细耐心的讲解和指导,对我的毕业设计的完成给与了巨大的帮助。

我也要感谢实验室的同学们,在三年的训练中,我们一起互相帮助,同甘共苦,攻坚克难,是我奋斗路上可靠的队友。愿我们的友谊地久天长,愿你们今后取得更高的成就。

另外,我还要感谢开源社区中无数勤劳无私的贡献者们,没有你们开发的众多的开源工具,本文的工作无法完成。在此感谢开源工作者,以及众多博主的技术博客,让我学到了很多。

最后,还要感谢我的家人,无论我身在哪里,家人都一如既往的支持鼓励着我,是我前进路上最坚强的后盾。

感谢所有支持帮助我的人,也祝各位同学前途无量。
\end{acknowledgement}

%%%%%%%%%%%%%%%%%%%%%%%%%%%%%%%%%%%%%%%%%%%%%%%%%%%%%%%%%%%%%%%%%%%%%%%%%%%%%%%
\end{document}
