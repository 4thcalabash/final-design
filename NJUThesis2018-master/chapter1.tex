\chapter{绪论}\label{chapter_introduction}

本章将介绍一些与本文研究相关研究背景,简要的总结前人的相关工作,并指出前人工作的不足之处。进而说明本文的主要贡献,以及本文在实现时面临的几个主要的挑战,最后说明本文的结构。
\section{研究背景}
在安卓应用中,服务(Services)和广播(Broadcast)得到了广泛的使用。服务可以在安卓应用的后台保持长期运行,提供诸如下载、数据更新等重要功能。然而,正因为服务长期运行于后台的特点,使其往往容易产生异常(Errors)。如果服务的编写人员缺少警惕性,服务中出现的错误(Bug)可能会导致诸多问题,严重者可能引起应用崩溃,甚至系统死机;广播是一种常被用来进行跨应用的通信的通信手段,应用在使用广播进行与系统或者与其他应用进行通讯时,应用需要编写广播接收器(Broadcast Receiver)。负责运行广播接收器的时应用的主线程种,因此在广播接收器中不适合进行耗时操作,通常会在广播接收器中启动服务来进行后续的处理,因此广播接收器也可能通过服务或者自身导致内存泄漏。

安卓应用中的内存泄露指资源(内存对象、句柄、服务等)将不再被使用,但却无法被安卓系统的垃圾回收器(Garbage Collector)所回收,同时也是服务中的一大类常见问题。服务如果出现内存泄露,将会导致内存使用量意外大幅度增加,进而使得系统效率降低,严重影响用户体验。

有一类服务被称为\textbf{公开服务},即指定了\textbf{“exported:true”}属性的服务。其他应用在满足一定条件时(满足权限要求等)可以启动应用的公开服务,因此内存泄露的问题将会变得更加复杂。

由于在安卓8及更高的版本下,安卓操作系统的“电池优化策略”\cite{pms}禁止跨应用启动后台服务\cite{android-service-limit},而这一方式在安卓7以及更早的版本中是可行的,因此在新版本(安卓8及更高)的安卓系统中,公开服务的内存泄漏检测方法与之前的方法\cite{jun2018lesdroid}有所差别,也正因为禁止跨应用启动后台服务,公开服务的内存泄漏问题也得到了很大的规避。

\section{相关工作}

Erika 等人在安卓8之前的版本中,编写了一个检测跨应用通信安全问题的工具\textbf{Com Droid}\cite{chin2011analyzing},文中阐述的方法对于跨应用测试具有指导意义。
\newline

刘洁瑞等人在2016年,针对安卓资源的内存泄漏问题,构造了\textbf{35}组标准测试集并使用\textbf{ResLeakBench}\cite{liu2016}进行了较为广泛的测试,该文所研究的系统资源泄露的场景,对本文研究的组件内存泄漏的场景有一定的借鉴意义。
\newline

在安卓8之前的版本中,跨应用启动服务这一行为是被允许的,南京大学的马骏等人安卓8之前的版本中,实现过一个公开服务(Exported Services)内存泄漏的检测工具LES Droid\cite{jun2018lesdroid},文中采用的方式分为四步:

\begin{enumerate}
	\item 使用apktool反编译工具\cite{apktool}对被测试应用进行反编译,获取被测试应用的安卓组件清单(AndroidManifest.xml)文件,解析获取应用中所有的公开服务的包名和类名。
	\item 将测试驱动应用、被测试应用通过adb安装到模拟器中,启动测试驱动应用。
	\item 测试驱动应用重复启动、关闭被测试的服务,在满足一定测试强度之后,导出被测试应用的堆镜像文件(.hprof files)。
	\item 基于MAT内存分析库\cite{mat}编写堆镜像文件的分析工具,自动检测内存泄漏并统计泄露的入口等。
\end{enumerate}

\label{pre-result}
文中的数据指出:在41537个被测试应用中,共在其中28662(69\%)个应用中检测出74831个服务,其中3934(13.7\%)个应用拥有公开服务。经过去重、安装测试以及应用商店评分筛选,有375个实际测试应用,最终通过不同的测试配置,最终检测到在18.7\%的应用中有16.8\%的服务存在内存泄漏问题。

\section{主要挑战}

本文面临的挑战主要有:

\begin{enumerate}
	\item 如何驱动测试流程自动化进行。
	\item 寻找办法能解决“电池管理限制”对测试流程和方法带来的多种限制。
	\item 如何自动化进行分析,统计出现内存泄露问题的组件清单。
\end{enumerate}

\section{现有研究的不足}

现有研究的不足主要体现在时间上的滞后:短时间内安卓版本经历了多次大的版本升级,而且这些升级对已有的测试方法(“电池管理限制”限制了很多跨应用通信的手段),以及被研究的测试对象(广播和服务的注册、启动、测试方法都有变化)都产生了一定的影响,但近年并没有新的研究跟进这些变化。

\section{本文贡献}

本文将会在最新的安卓版本(安卓10)上对服务和广播进行内存泄漏的检测和测试。探索了一种解除电源管理限制的方法。分析实验数据,评价“电池管理限制”带来的影响。

\section{本文主要工作}
本文旨在探索一套适用于安卓10版本的公开服务和静态注册广播接收器的内存泄漏检测方法。主要工作如下:
\newline
\begin{enumerate}
\item 找到在安卓8以上版本的安卓系统上可行的跨应用测试方法。

\item 对桩应用上进行测试,并能发现所有泄露。

\item 在应用商店中下载真实应用,进行自动化测试分析实验结果。

\end{enumerate}
\section{本文结构}
本文的各章节组织结构如下:
第一章\textbf{绪论\ref{chapter_introduction}}简要的说明了本文的研究对象,和采用的解决办法,并总结了相关论文的工作等。第二章\textbf{背景\ref{chapter_background}}详细介绍了安卓广播和服务的注册、启动、生命周期等基础知识,并且对安卓组件内存泄漏的原理做了举例。第三章\textbf{技术方案\ref{chapter_system}}中具体地解释了自动化检测工具的所有流程和实现细节,并通过安卓源码发现了后台启动服务的方法(后台启动服务默认为不允许的行为)。第四章\textbf{实验\ref{chapter_experiment}}进行了仿真测试与真实测试,仿真测试目的在于验证自动检测工具的正确性和可行性,真实测试从安卓应用市场中下载了真实应用,使用本文的自动检测工具进行了测试。第五章\textbf{结论\ref{chapter_conclusion}}总结了本文的主要工作,并通过数据对比发现在新版本的安卓系统中,应用的组件内存泄漏问题依然大量存在。